\documentclass[10pt]{article}
\usepackage[
    landscape,
    twocolumn,
    a4paper,
    left=0.7in,
    right=0.7in,
    bottom=1in,
    top=0.7in]{geometry}
\usepackage{fontspec}
\setmainfont{Brill}
\usepackage[abbrevs=none,refmode=latex]{expex-acro}
\lingset{belowglpreambleskip=-1ex,everyglpreamble=\itshape,aboveglftskip=-0.5ex,aboveexskip=-0.7ex,belowexskip=-1ex}
\usepackage{booktabs}
\usepackage[style=authoryear]{biblatex}
\addbibresource{/home/florianm/Dropbox/research/cariban/yawarana/yaw_cldf/cldf/sources.bib}
\def\tightlist{\setlength\itemsep{0em}}
\usepackage{hyperref}
\usepackage[capitalise]{cleveref}
\begin{document}
\begin{center}
\Large \bfseries {Ambiguities in parsing Yawarana data: practical and theoretical considerations}
\end{center}

\section{Introduction \label{sec:intro}}

\begin{itemize}
\tightlist
\item
  Yawarana
  (\href{https://glottolog.org/resource/languoid/id/yaba1248}{yaba1248})
  is an underdescribed Cariban language of Venezuela.
\item
  \textcite{matter2022uniparser}
\end{itemize}

\printbibliography

\end{document}